\PassOptionsToPackage{unicode=true}{hyperref} % options for packages loaded elsewhere
\PassOptionsToPackage{hyphens}{url}
%
\documentclass[]{article}
\usepackage{lmodern}
\usepackage{amssymb,amsmath}
\usepackage{ifxetex,ifluatex}
\usepackage{fixltx2e} % provides \textsubscript
\ifnum 0\ifxetex 1\fi\ifluatex 1\fi=0 % if pdftex
  \usepackage[T1]{fontenc}
  \usepackage[utf8]{inputenc}
  \usepackage{textcomp} % provides euro and other symbols
\else % if luatex or xelatex
  \usepackage{unicode-math}
  \defaultfontfeatures{Ligatures=TeX,Scale=MatchLowercase}
\fi
% use upquote if available, for straight quotes in verbatim environments
\IfFileExists{upquote.sty}{\usepackage{upquote}}{}
% use microtype if available
\IfFileExists{microtype.sty}{%
\usepackage[]{microtype}
\UseMicrotypeSet[protrusion]{basicmath} % disable protrusion for tt fonts
}{}
\IfFileExists{parskip.sty}{%
\usepackage{parskip}
}{% else
\setlength{\parindent}{0pt}
\setlength{\parskip}{6pt plus 2pt minus 1pt}
}
\usepackage{hyperref}
\hypersetup{
            pdfborder={0 0 0},
            breaklinks=true}
\urlstyle{same}  % don't use monospace font for urls
\setlength{\emergencystretch}{3em}  % prevent overfull lines
\providecommand{\tightlist}{%
  \setlength{\itemsep}{0pt}\setlength{\parskip}{0pt}}
\setcounter{secnumdepth}{0}
% Redefines (sub)paragraphs to behave more like sections
\ifx\paragraph\undefined\else
\let\oldparagraph\paragraph
\renewcommand{\paragraph}[1]{\oldparagraph{#1}\mbox{}}
\fi
\ifx\subparagraph\undefined\else
\let\oldsubparagraph\subparagraph
\renewcommand{\subparagraph}[1]{\oldsubparagraph{#1}\mbox{}}
\fi

% set default figure placement to htbp
\makeatletter
\def\fps@figure{htbp}
\makeatother


\date{}

\begin{document}

\hypertarget{api-test-plan-for-sams}{%
\section{API Test Plan for SAMS}\label{api-test-plan-for-sams}}

\hypertarget{authors}{%
\subsection{Authors}\label{authors}}

Group 13 (Eternal Blue)

Authors:

\begin{itemize}
\tightlist
\item
  Aaditya Agrawal (19CS10003)
\item
  Debanjan Saha (19CS30014)
\item
  Deep Majumder (19CS30015)
\end{itemize}

\hypertarget{expenditure}{%
\subsection{Expenditure}\label{expenditure}}

\hypertarget{post-an-expenditure}{%
\subsubsection{POST an Expenditure}\label{post-an-expenditure}}

Tests the POST method at API endpoint:
\texttt{http://localhost:8080/expenditures/}

Test Data:

\begin{enumerate}
\def\labelenumi{\arabic{enumi}.}
\tightlist
\item
  Expenditure testObject:

  \begin{itemize}
  \tightlist
  \item
    amount: 212.50
  \item
    reason: ``Electricity Bill''
  \item
    showId: 404
  \end{itemize}
\item
  AccountantId: 30014
\end{enumerate}

Successful \texttt{POST}:

\begin{itemize}
\tightlist
\item
  An object of ExpenditureCreation is created from testObject and
  AccountantId and a \texttt{POST} call is made.
\item
  A new expenditure is created and saved in the database and it returns
  an Expenditure object.
\item
  The returned object \textbf{must not} be null.
\item
  The returned object must have the same attributes as that of given
  testObject.
\end{itemize}

\hypertarget{get-all-expenditures}{%
\subsubsection{GET all Expenditures}\label{get-all-expenditures}}

Tests the GET method at API endpoint:
\texttt{http://localhost:8080/expenditures/}

Test Data:

\begin{enumerate}
\def\labelenumi{\arabic{enumi}.}
\tightlist
\item
  Expenditure testObject1:

  \begin{itemize}
  \tightlist
  \item
    amount: 212.50
  \item
    reason: ``Electricity Bill''
  \item
    showId: 404
  \end{itemize}
\item
  Expenditure testObject2:

  \begin{itemize}
  \tightlist
  \item
    amount: 1500.00
  \item
    reason: ``AC Repairing''
  \item
    showId: 404
  \end{itemize}
\item
  Expenditure testObject3:

  \begin{itemize}
  \tightlist
  \item
    amount: 1299.99
  \item
    reason: ``Payment to Software Developer''
  \item
    showId: 404
  \end{itemize}
\item
  Expenditure testObject4:

  \begin{itemize}
  \tightlist
  \item
    amount: 1200.00
  \item
    reason: ``Payment to Artist''
  \item
    showId: 100
  \end{itemize}
\item
  AccountantId: 30014
\end{enumerate}

Successful \texttt{GET}:

\begin{itemize}
\tightlist
\item
  4 objects of ExpenditureCreation are created from testObjects and
  AccountantId and 4 \texttt{POST} calls are made.
\item
  On \texttt{GET} call, a List of Expenditure is returned.
\item
  The returned list \textbf{must not} be empty and should have size = 4.
\end{itemize}

\hypertarget{get-expenditure-by-id}{%
\subsubsection{GET Expenditure by Id}\label{get-expenditure-by-id}}

Tests the GET method at API endpoint:
\texttt{http://localhost:8080/expenditures/\{id\}}

Test Data:

\begin{enumerate}
\def\labelenumi{\arabic{enumi}.}
\tightlist
\item
  Expenditure testObject:

  \begin{itemize}
  \tightlist
  \item
    amount: 212.50
  \item
    reason: ``Electricity Bill''
  \item
    showId: 404
  \end{itemize}
\item
  AccountantId: 30014
\end{enumerate}

Successful \texttt{GET}:

\begin{itemize}
\tightlist
\item
  An objects of ExpenditureCreation is created from testObjects and
  AccountantId and a \texttt{POST} call is made.
\item
  On \texttt{GET} call with the id of the object returned from the
  \texttt{POST} call, another object of Expenditure is returned.
\item
  The returned object \textbf{must not} be null.
\item
  The objects returned from \texttt{POST} and \texttt{GET} calls must
  have same attributes as given in test data.
\end{itemize}

\hypertarget{get-expenditure-by-show-id}{%
\subsubsection{GET Expenditure by Show
Id}\label{get-expenditure-by-show-id}}

Tests the GET method at API endpoint:
\texttt{http://localhost:8080/expenditures/by\_show/\{showId\}}

Test Data:

\begin{enumerate}
\def\labelenumi{\arabic{enumi}.}
\tightlist
\item
  Expenditure testObject1:

  \begin{itemize}
  \tightlist
  \item
    amount: 1212.50
  \item
    reason: ``AC Repairing''
  \item
    showId: 404
  \end{itemize}
\item
  Expenditure testObject2:

  \begin{itemize}
  \tightlist
  \item
    amount: 1299.99
  \item
    reason: ``Payment to Software Developer''
  \item
    showId: 404
  \end{itemize}
\item
  Expenditure testObject3:

  \begin{itemize}
  \tightlist
  \item
    amount: 1200.00
  \item
    reason: ``Payment to Artist''
  \item
    showId: 100
  \end{itemize}
\item
  AccountantId: 30014
\end{enumerate}

Successful \texttt{GET}:

\begin{itemize}
\tightlist
\item
  3 objects of ExpenditureCreation are created from testObjects and
  AccountantId and 3 \texttt{POST} calls are made.
\item
  On \texttt{GET} call with the showId 404, a List of Expenditure is
  returned.
\item
  The returned object \textbf{must not} be empty and should have size 2
  as two testObjects have showId = 404.
\item
  On \texttt{GET} call with the showId 100, another List of Expenditure
  is returned.
\item
  The returned object \textbf{must not} be empty and should have size 1
  as only one testObject has showId = 100.
\end{itemize}

\hypertarget{ticket}{%
\subsection{Ticket}\label{ticket}}

\hypertarget{get---tickets}{%
\subsubsection{GET - /tickets}\label{get---tickets}}

Get all the tickets.

\begin{itemize}
\tightlist
\item
  The database is pre-populated with some ticket instances.
\item
  The request is performed
\item
  The response status must be 200
\item
  The tickets are deserialized from the response JSON.
\item
  The obtained list and the original list is compared to be same.
\end{itemize}

\hypertarget{post---tickets}{%
\subsubsection{POST - /tickets}\label{post---tickets}}

Create a new ticket.

\begin{itemize}
\tightlist
\item
  A new Ticket object is created
\item
  Combined with salesperson ID, a TicketCreation object is created
\item
  Using the above object, the request is performed
\item
  The response status must be 200
\item
  The created ticket is deserialized from the response JSON
\item
  The obtained ticket is compared to be same as the original ticket
\end{itemize}

\hypertarget{get---ticketsid}{%
\subsubsection{GET - /tickets/\{id\}}\label{get---ticketsid}}

Gets a particular ticket

\begin{itemize}
\tightlist
\item
  Take the id of a ticket that has been pre-populated
\item
  Pass the id as param and perform the request
\item
  Deserialize the Ticket from the response JSON
\item
  Compare the ticket obtained to be same as the original ticket
\end{itemize}

\hypertarget{delete---ticketsid}{%
\subsubsection{DELETE - /tickets/\{id\}}\label{delete---ticketsid}}

Deletes a particular ticket

\begin{itemize}
\tightlist
\item
  Take the id of a ticket that has been pre-populated
\item
  Pass the id as param and perform the request
\item
  Deserialize the response JSON to get DeleteResult
\item
  Check isDeleted() to be \emph{true} and refundAmount() to be
  \emph{price - 5}.
\item
  Call GET /tickets to get all the tickets.
\item
  Check that the deleted ticket is not present in the list
\end{itemize}

\hypertarget{get---ticketsby_userid}{%
\subsubsection{GET -
/tickets/by\_user/\{id\}}\label{get---ticketsby_userid}}

Gets all tickets of a user

\begin{itemize}
\tightlist
\item
  Take the id of the User object whose tickets have been pre-populated
\item
  Pass the id as params and perform the request
\item
  Deserialize the response JSON to get the list of Tickets
\item
  Check that this is same as the orignal list of Tickets
\end{itemize}

\hypertarget{get---ticketsby_showid}{%
\subsubsection{GET -
/tickets/by\_show/\{id\}}\label{get---ticketsby_showid}}

Gets all tickets of a show

\begin{itemize}
\tightlist
\item
  Take the id of the Show object whose tickets have been pre-populated
\item
  Pass the id as params and perform the request
\item
  Deserialize the response JSON to get the list of Tickets
\item
  Check that this is same as the original list of Tickets
\end{itemize}

\hypertarget{transaction}{%
\subsection{Transaction}\label{transaction}}

\hypertarget{get---transactions}{%
\subsubsection{GET - /transactions}\label{get---transactions}}

Get all transactions

\begin{itemize}
\tightlist
\item
  Pre-populate the repository with transactions
\item
  Perform the request
\item
  Deserialize the response JSON to get the list of Transactions
\item
  Check that this is same as the original list of Transactions
\end{itemize}

\hypertarget{get---transactionsid}{%
\subsubsection{GET - /transactions/\{id\}}\label{get---transactionsid}}

Get one particular transactions

\begin{itemize}
\tightlist
\item
  Take the id of a transaction that has been pre-populated
\item
  Perform the request with the id parameter
\item
  Deserialize the JSON to get the Transaction
\item
  Check that this is same as the original Transaction
\end{itemize}

\hypertarget{get---transactionsby_showid}{%
\subsubsection{GET -
/transactions/by\_show/\{id\}}\label{get---transactionsby_showid}}

Get all transactions regarding a show

\begin{itemize}
\tightlist
\item
  Pre-populate the repository with some transactions
\item
  Perform the request with the id of a show whose transaction is present
\item
  Deserialize the JSON to get the list of Transactions
\item
  Check that this is the same as the original transactions, filtered
  over show id
\end{itemize}

\hypertarget{get---transactionsby_salespersonid}{%
\subsubsection{GET -
/transactions/by\_salesperson/\{id\}}\label{get---transactionsby_salespersonid}}

Get all transactions initiated by a salesperson

\begin{itemize}
\tightlist
\item
  Pre-populate the repository with some transactions
\item
  Perform the request with the id of a salesperson who has initiated
  some transactions
\item
  Deserialize the JSON to get the list of Transactions
\item
  Check that this is the same as the original transactions, filtered
  over initiator id and type as Salesperson
\end{itemize}

\hypertarget{get---transactionsby_yearyear}{%
\subsubsection{GET -
/transactions/by\_year/\{year\}}\label{get---transactionsby_yearyear}}

Get all transactions of a year

\begin{itemize}
\tightlist
\item
  Pre-populate the repository with some transactions
\item
  Perform the request with a year whose transaction is present
\item
  Deserialize the JSON to get the list of Transactions
\item
  Check that this is the same as the original transactions, filtered
  over year
\end{itemize}

\hypertarget{user}{%
\subsection{User}\label{user}}

Before every test is run, the database is cleared.

\hypertarget{post-an-user}{%
\subsubsection{POST an User}\label{post-an-user}}

Tests the POST method at endpoint \texttt{http://localhost:8080/users/}

This accepts an user and stores it in the database.

\begin{itemize}
\tightlist
\item
  An instance of user is created and a \texttt{POST} call is made.
\item
  The controller layer returns the instance back if user was
  successfully saved in the DataBase.
\item
  The returned user is extracted. Their corresponding fields must match.
\end{itemize}

\hypertarget{get-an-user-by-id}{%
\subsubsection{GET an User by Id}\label{get-an-user-by-id}}

Tests the GET method at endpoint
\texttt{http://localhost:8080/users/\{userId\}}

This accepts an user id and returns the instance of user if it exists.

\begin{itemize}
\tightlist
\item
  An instance of user is created and a \texttt{POST} call is made.
\item
  The repository is queried, by doing a \texttt{GET} call, for an user
  instance having the same id as that of the instance which was POSTed.
\item
  The returned user is extracted. The corresponding fields must match.
\end{itemize}

\hypertarget{get-an-user-by-login-detail---userslogin}{%
\subsubsection{GET an User by login detail -
/users/login}\label{get-an-user-by-login-detail---userslogin}}

Tests the GET method at endpoint
\texttt{http://localhost:8080/users/login}

This takes username and password of an user, and returns the instance of
user if it exists.

\begin{itemize}
\tightlist
\item
  An instance of user is created and a \texttt{POST} call is made.
\item
  The repository is queried, by doing a \texttt{GET} call, for an user
  instance having the same username and password as that of the instance
  which was POSTed.
\item
  The returned user is extracted. The corresponding fields must match.
\end{itemize}

\hypertarget{get-all-users}{%
\subsubsection{GET all Users}\label{get-all-users}}

Tests the GET method at endpoint \texttt{http://localhost:8080/users}

This returns list of all the users from the databases.

\begin{itemize}
\tightlist
\item
  A few users are added to the database by doing \texttt{POST} calls.
\item
  A \texttt{GET} call is made to retrieve list of all users and then
  size of list of users is verified.
\end{itemize}

\hypertarget{delete-user-by-id}{%
\subsubsection{DELETE user By Id}\label{delete-user-by-id}}

Tests the DELETE method at endpoint
\texttt{http://localhost:8080/users/\{userId\}}

This accepts an user id and deletes the user from the database.

\begin{itemize}
\tightlist
\item
  An user is added to the database by doing a \texttt{POST} call.
\item
  A \texttt{DELETE} call is made to the database with the id of the user
  just created.
\item
  A \texttt{GET} call is made to make sure that user doesn't exist
  anymore.
\end{itemize}

\hypertarget{shows}{%
\subsection{Shows}\label{shows}}

Before every test is run, the database is cleared.

\hypertarget{post-a-show}{%
\subsubsection{POST a Show}\label{post-a-show}}

Tests the POST method at endpoint \texttt{http://localhost:8080/shows}

This accepts an show and stores it in the database.

\begin{itemize}
\tightlist
\item
  An instance of show is created and a \texttt{POST} call is made.
\item
  The controller layer returns the instance back if show was
  successfully saved in the DataBase.
\item
  The returned show is extracted. Their corresponding fields must match.
\end{itemize}

\hypertarget{get-a-show-by-id}{%
\subsubsection{GET a show by Id}\label{get-a-show-by-id}}

Tests the GET method at endpoint
\texttt{http://localhost:8080/shows/\{showId\}}

This accepts a show id and returns the instance of show if it exists.

\begin{itemize}
\tightlist
\item
  An instance of show is created and a \texttt{POST} call is made.
\item
  The repository is queried, by doing a \texttt{GET} call, for a show
  instance having the same id as that of the instance which was POSTed.
\item
  The returned show is extracted. The corresponding fields must match.
\end{itemize}

\hypertarget{get-all-shows}{%
\subsubsection{GET all Shows}\label{get-all-shows}}

Tests the GET method at endpoint \texttt{http://localhost:8080/shows}

This returns list of all the shows from the databases.

\begin{itemize}
\tightlist
\item
  A few shows are added to the database by doing \texttt{POST} calls.
\item
  A \texttt{GET} call is made to retrieve list of all shows and then
  size of list of shows is verified.
\end{itemize}

\end{document}
